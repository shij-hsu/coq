\documentclass[12pt]{report}
\usepackage[]{inputenc}
\usepackage[T1]{fontenc}
\usepackage{fullpage}
\usepackage{coqdoc}
\usepackage{amsmath,amssymb}
\usepackage{url}
\begin{document}
%%%%%%%%%%%%%%%%%%%%%%%%%%%%%%%%%%%%%%%%%%%%%%%%%%%%%%%%%%%%%%%%%
%% This file has been automatically generated with the command
%% coqdoc --latex -s ex.v 
%%%%%%%%%%%%%%%%%%%%%%%%%%%%%%%%%%%%%%%%%%%%%%%%%%%%%%%%%%%%%%%%%
\begin{coqdoccode}
\coqdocemptyline
\end{coqdoccode}
\chapter{From Inductive Types} \begin{coqdoccode}
\coqdocemptyline
\coqdocnoindent
\coqdockw{Require} \coqdockw{Import} \coqexternalref{}{http://coq.inria.fr/distrib/8.5pl1/stdlib/Coq.Setoids.Setoid}{\coqdoclibrary{Coq.Setoids.Setoid}}.\coqdoceol
\coqdocemptyline
\end{coqdoccode}
\begin{enumerate}


\item Define an inductive type \coqdocvar{truth} with three constructors, \coqdocvar{Yes}, \coqdocvar{No}, and \coqdocvar{Maybe}. \coqdocvar{Yes} stands for certain truth, \coqdocvar{No} for certain falsehood, and \coqdocvar{Maybe} for an unknown situation.  Define ``not,'' ``and,'' and ``or'' for this replacement boolean algebra.  Prove that your implementation of ``and'' is commutative and distributes over your implementation of ``or.''
\begin{coqdoccode}
\coqdocemptyline
\coqdocnoindent
\coqdockw{Module} \coqdef{ex.ex1}{ex1}{\coqdocmodule{ex1}}.\coqdoceol
\coqdocnoindent
\coqdockw{Inductive} \coqdef{ex.ex1.truth}{truth}{\coqdocinductive{truth}} : \coqdockw{Type} := \coqdef{ex.ex1.Yes}{Yes}{\coqdocconstructor{Yes}} \ensuremath{|} \coqdef{ex.ex1.No}{No}{\coqdocconstructor{No}} \ensuremath{|} \coqdef{ex.ex1.Maybe}{Maybe}{\coqdocconstructor{Maybe}}.\coqdoceol
\coqdocemptyline
\coqdocnoindent
\coqdockw{Definition} \coqdef{ex.ex1.not}{not}{\coqdocdefinition{not}} (\coqdocvar{a} : \coqref{ex.ex1.truth}{\coqdocinductive{truth}}) : \coqref{ex.ex1.truth}{\coqdocinductive{truth}} :=\coqdoceol
\coqdocindent{1.00em}
\coqdockw{match} \coqdocvariable{a} \coqdockw{with}\coqdoceol
\coqdocindent{1.00em}
\ensuremath{|} \coqref{ex.ex1.Yes}{\coqdocconstructor{Yes}} \ensuremath{\Rightarrow} \coqref{ex.ex1.No}{\coqdocconstructor{No}}\coqdoceol
\coqdocindent{1.00em}
\ensuremath{|} \coqref{ex.ex1.No}{\coqdocconstructor{No}} \ensuremath{\Rightarrow} \coqref{ex.ex1.Yes}{\coqdocconstructor{Yes}}\coqdoceol
\coqdocindent{1.00em}
\ensuremath{|} \coqref{ex.ex1.Maybe}{\coqdocconstructor{Maybe}} \ensuremath{\Rightarrow} \coqref{ex.ex1.Maybe}{\coqdocconstructor{Maybe}}\coqdoceol
\coqdocindent{1.00em}
\coqdockw{end}.\coqdoceol
\coqdocemptyline
\coqdocnoindent
\coqdockw{Check} \coqref{ex.ex1.not}{\coqdocdefinition{not}} \coqref{ex.ex1.Yes}{\coqdocconstructor{Yes}}.\coqdoceol
\coqdocemptyline
\coqdocnoindent
\coqdockw{Definition} \coqdef{ex.ex1.and}{and}{\coqdocdefinition{and}} (\coqdocvar{a} \coqdocvar{b} : \coqref{ex.ex1.truth}{\coqdocinductive{truth}}) : \coqref{ex.ex1.truth}{\coqdocinductive{truth}} :=\coqdoceol
\coqdocnoindent
\coqdockw{match} \coqdocvariable{a} \coqdockw{with}\coqdoceol
\coqdocnoindent
\ensuremath{|} \coqref{ex.ex1.Yes}{\coqdocconstructor{Yes}} \ensuremath{\Rightarrow} \coqdocvariable{b}\coqdoceol
\coqdocnoindent
\ensuremath{|} \coqref{ex.ex1.No}{\coqdocconstructor{No}} \ensuremath{\Rightarrow} \coqdockw{match} \coqdocvariable{b} \coqdockw{with}\coqdoceol
\coqdocindent{3.50em}
\ensuremath{|} \coqref{ex.ex1.Maybe}{\coqdocconstructor{Maybe}} \ensuremath{\Rightarrow} \coqref{ex.ex1.Maybe}{\coqdocconstructor{Maybe}}\coqdoceol
\coqdocindent{3.50em}
\ensuremath{|} \coqdocvar{\_} \ensuremath{\Rightarrow} \coqref{ex.ex1.No}{\coqdocconstructor{No}}\coqdoceol
\coqdocindent{3.50em}
\coqdockw{end}\coqdoceol
\coqdocnoindent
\ensuremath{|} \coqref{ex.ex1.Maybe}{\coqdocconstructor{Maybe}} \ensuremath{\Rightarrow} \coqref{ex.ex1.Maybe}{\coqdocconstructor{Maybe}}\coqdoceol
\coqdocnoindent
\coqdockw{end}.\coqdoceol
\coqdocemptyline
\coqdocnoindent
\coqdockw{Definition} \coqdef{ex.ex1.or}{or}{\coqdocdefinition{or}} (\coqdocvar{a} \coqdocvar{b} : \coqref{ex.ex1.truth}{\coqdocinductive{truth}}) : \coqref{ex.ex1.truth}{\coqdocinductive{truth}} :=\coqdoceol
\coqdocindent{1.00em}
\coqdockw{match} \coqdocvariable{a} \coqdockw{with}\coqdoceol
\coqdocindent{1.00em}
\ensuremath{|} \coqref{ex.ex1.Yes}{\coqdocconstructor{Yes}} \ensuremath{\Rightarrow} \coqdockw{match} \coqdocvariable{b} \coqdockw{with}\coqdoceol
\coqdocindent{5.00em}
\ensuremath{|} \coqref{ex.ex1.Maybe}{\coqdocconstructor{Maybe}} \ensuremath{\Rightarrow} \coqref{ex.ex1.Maybe}{\coqdocconstructor{Maybe}}\coqdoceol
\coqdocindent{5.00em}
\ensuremath{|} \coqdocvar{\_} \ensuremath{\Rightarrow} \coqref{ex.ex1.Yes}{\coqdocconstructor{Yes}}\coqdoceol
\coqdocindent{5.00em}
\coqdockw{end}\coqdoceol
\coqdocindent{1.00em}
\ensuremath{|} \coqref{ex.ex1.No}{\coqdocconstructor{No}} \ensuremath{\Rightarrow} \coqdocvariable{b}\coqdoceol
\coqdocindent{1.00em}
\ensuremath{|} \coqref{ex.ex1.Maybe}{\coqdocconstructor{Maybe}} \ensuremath{\Rightarrow} \coqref{ex.ex1.Maybe}{\coqdocconstructor{Maybe}}\coqdoceol
\coqdocindent{1.00em}
\coqdockw{end}.\coqdoceol
\coqdocemptyline
\coqdocnoindent
\coqdockw{Lemma} \coqdef{ex.ex1.and comm}{and\_comm}{\coqdoclemma{and\_comm}} : \coqdockw{\ensuremath{\forall}} (\coqdocvar{a} \coqdocvar{b} : \coqref{ex.ex1.truth}{\coqdocinductive{truth}}), \coqref{ex.ex1.and}{\coqdocdefinition{and}} \coqdocvariable{a} \coqdocvariable{b} \coqexternalref{:type scope:x '=' x}{http://coq.inria.fr/distrib/8.5pl1/stdlib/Coq.Init.Logic}{\coqdocnotation{=}} \coqref{ex.ex1.and}{\coqdocdefinition{and}} \coqdocvariable{b} \coqdocvariable{a}.\coqdoceol
\coqdocindent{1.00em}
\coqdoctac{intros}; \coqdoctac{destruct} \coqdocvar{a}, \coqdocvar{b}; \coqdoctac{auto}. \coqdockw{Qed}.\coqdoceol
\coqdocnoindent
\coqdockw{Lemma} \coqdef{ex.ex1.or comm}{or\_comm}{\coqdoclemma{or\_comm}} : \coqdockw{\ensuremath{\forall}} (\coqdocvar{a} \coqdocvar{b} : \coqref{ex.ex1.truth}{\coqdocinductive{truth}}), \coqref{ex.ex1.or}{\coqdocdefinition{or}} \coqdocvariable{a} \coqdocvariable{b} \coqexternalref{:type scope:x '=' x}{http://coq.inria.fr/distrib/8.5pl1/stdlib/Coq.Init.Logic}{\coqdocnotation{=}} \coqref{ex.ex1.or}{\coqdocdefinition{or}} \coqdocvariable{b} \coqdocvariable{a}.\coqdoceol
\coqdocindent{1.00em}
\coqdoctac{intros}; \coqdoctac{destruct} \coqdocvar{a}, \coqdocvar{b}; \coqdoctac{auto}. \coqdockw{Qed}.\coqdoceol
\coqdocnoindent
\coqdockw{Lemma} \coqdef{ex.ex1.or distr}{or\_distr}{\coqdoclemma{or\_distr}} : \coqdockw{\ensuremath{\forall}} (\coqdocvar{a} \coqdocvar{b} \coqdocvar{c} : \coqref{ex.ex1.truth}{\coqdocinductive{truth}}), \coqref{ex.ex1.or}{\coqdocdefinition{or}} (\coqref{ex.ex1.and}{\coqdocdefinition{and}} \coqdocvariable{a} \coqdocvariable{b}) \coqdocvariable{c} \coqexternalref{:type scope:x '=' x}{http://coq.inria.fr/distrib/8.5pl1/stdlib/Coq.Init.Logic}{\coqdocnotation{=}} \coqref{ex.ex1.and}{\coqdocdefinition{and}} (\coqref{ex.ex1.or}{\coqdocdefinition{or}} \coqdocvariable{a} \coqdocvariable{c}) (\coqref{ex.ex1.or}{\coqdocdefinition{or}} \coqdocvariable{b} \coqdocvariable{c}).\coqdoceol
\coqdocindent{1.00em}
\coqdoctac{intros}; \coqdoctac{destruct} \coqdocvar{a}, \coqdocvar{b}, \coqdocvar{c}; \coqdoctac{auto}. \coqdockw{Qed}.\coqdoceol
\coqdocemptyline
\coqdocnoindent
\coqdockw{End} \coqref{ex.ex1}{\coqdocmodule{ex1}}.\coqdoceol
\coqdocemptyline
\end{coqdoccode}
\item Define an inductive type \coqdocvar{slist} that implements lists with support for constant-time concatenation.  This type should be polymorphic in a choice of type for data values in lists.  The type \coqdocvar{slist} should have three constructors, for empty lists, singleton lists, and concatenation.  Define a function \coqdocvar{flatten} that converts \coqdocvar{slist}s to \coqdocvar{list}s.  (You will want to run \coqdockw{Require} \coqdockw{Import} \coqdocconstructor{List}. to bring list definitions into scope.)  Finally, prove that \coqdocvar{flatten} distributes over concatenation, where the two sides of your quantified equality will use the \coqdocvar{slist} and \coqdocvar{list} versions of concatenation, as appropriate.  Recall from Chapter 2 that the infix operator ++ is syntactic sugar for the \coqdocvar{list} concatenation function \coqdocvar{app}.
 \begin{coqdoccode}
\coqdocemptyline
\coqdocnoindent
\coqdockw{Module} \coqdef{ex.ex2}{ex2}{\coqdocmodule{ex2}}.\coqdoceol
\coqdocnoindent
\coqdockw{Require} \coqdockw{Import} \coqexternalref{}{http://coq.inria.fr/distrib/8.5pl1/stdlib/Coq.Lists.List}{\coqdoclibrary{List}}.\coqdoceol
\coqdocnoindent
\coqdockw{Set Implicit Arguments}.\coqdoceol
\coqdocnoindent
\coqdockw{Inductive} \coqdef{ex.ex2.slist}{slist}{\coqdocinductive{slist}} (\coqdocvar{X} : \coqdockw{Type}) : \coqdockw{Type}:=\coqdoceol
\coqdocnoindent
\ensuremath{|} \coqdef{ex.ex2.s nil}{s\_nil}{\coqdocconstructor{s\_nil}} : \coqref{ex.slist}{\coqdocinductive{slist}} \coqdocvar{X}\coqdoceol
\coqdocnoindent
\ensuremath{|} \coqdef{ex.ex2.s singleton}{s\_singleton}{\coqdocconstructor{s\_singleton}} : \coqdocvar{X} \coqexternalref{:type scope:x '->' x}{http://coq.inria.fr/distrib/8.5pl1/stdlib/Coq.Init.Logic}{\coqdocnotation{\ensuremath{\rightarrow}}} \coqref{ex.slist}{\coqdocinductive{slist}} \coqdocvar{X}\coqdoceol
\coqdocnoindent
\ensuremath{|} \coqdef{ex.ex2.s cons}{s\_cons}{\coqdocconstructor{s\_cons}} : \coqref{ex.slist}{\coqdocinductive{slist}} \coqdocvar{X} \coqexternalref{:type scope:x '->' x}{http://coq.inria.fr/distrib/8.5pl1/stdlib/Coq.Init.Logic}{\coqdocnotation{\ensuremath{\rightarrow}}} \coqref{ex.slist}{\coqdocinductive{slist}} \coqdocvar{X} \coqexternalref{:type scope:x '->' x}{http://coq.inria.fr/distrib/8.5pl1/stdlib/Coq.Init.Logic}{\coqdocnotation{\ensuremath{\rightarrow}}} \coqref{ex.slist}{\coqdocinductive{slist}} \coqdocvar{X}.\coqdoceol
\coqdocemptyline
\coqdocnoindent
\coqdockw{Fixpoint} \coqdef{ex.ex2.flattern}{flattern}{\coqdocdefinition{flattern}} (\coqdocvar{X} : \coqdockw{Type}) (\coqdocvar{sl} : \coqref{ex.ex2.slist}{\coqdocinductive{slist}} \coqdocvariable{X}) : \coqexternalref{list}{http://coq.inria.fr/distrib/8.5pl1/stdlib/Coq.Init.Datatypes}{\coqdocinductive{list}} \coqdocvariable{X} :=\coqdoceol
\coqdocindent{1.00em}
\coqdockw{match} \coqdocvariable{sl} \coqdockw{with}\coqdoceol
\coqdocindent{1.00em}
\ensuremath{|} @\coqdocvar{s\_nil} \coqdocvar{\_} \ensuremath{\Rightarrow} \coqexternalref{nil}{http://coq.inria.fr/distrib/8.5pl1/stdlib/Coq.Init.Datatypes}{\coqdocconstructor{nil}}\coqdoceol
\coqdocindent{1.00em}
\ensuremath{|} \coqref{ex.ex2.s singleton}{\coqdocconstructor{s\_singleton}} \coqdocvar{a} \ensuremath{\Rightarrow} \coqdocvar{a}\coqexternalref{:list scope:x '::' x}{http://coq.inria.fr/distrib/8.5pl1/stdlib/Coq.Init.Datatypes}{\coqdocnotation{::}}\coqexternalref{nil}{http://coq.inria.fr/distrib/8.5pl1/stdlib/Coq.Init.Datatypes}{\coqdocconstructor{nil}}\coqdoceol
\coqdocindent{1.00em}
\ensuremath{|} \coqref{ex.ex2.s cons}{\coqdocconstructor{s\_cons}} \coqdocvar{sl1} \coqdocvar{sl2} \ensuremath{\Rightarrow} \coqexternalref{:list scope:x '++' x}{http://coq.inria.fr/distrib/8.5pl1/stdlib/Coq.Init.Datatypes}{\coqdocnotation{(}}\coqref{ex.flattern}{\coqdocdefinition{flattern}} \coqdocvar{sl1}\coqexternalref{:list scope:x '++' x}{http://coq.inria.fr/distrib/8.5pl1/stdlib/Coq.Init.Datatypes}{\coqdocnotation{)}} \coqexternalref{:list scope:x '++' x}{http://coq.inria.fr/distrib/8.5pl1/stdlib/Coq.Init.Datatypes}{\coqdocnotation{++}} \coqexternalref{:list scope:x '++' x}{http://coq.inria.fr/distrib/8.5pl1/stdlib/Coq.Init.Datatypes}{\coqdocnotation{(}}\coqref{ex.flattern}{\coqdocdefinition{flattern}} \coqdocvar{sl2}\coqexternalref{:list scope:x '++' x}{http://coq.inria.fr/distrib/8.5pl1/stdlib/Coq.Init.Datatypes}{\coqdocnotation{)}}\coqdoceol
\coqdocindent{1.00em}
\coqdockw{end}.\coqdoceol
\coqdocnoindent
\coqdockw{Fixpoint} \coqdef{ex.ex2.s app}{s\_app}{\coqdocdefinition{s\_app}} (\coqdocvar{X} : \coqdockw{Type}) (\coqdocvar{s1} \coqdocvar{s2} : \coqref{ex.ex2.slist}{\coqdocinductive{slist}} \coqdocvariable{X}) : \coqref{ex.ex2.slist}{\coqdocinductive{slist}} \coqdocvariable{X}:=\coqdoceol
\coqdocindent{1.00em}
\coqdockw{match} \coqdocvariable{s1} \coqdockw{with}\coqdoceol
\coqdocindent{1.00em}
\ensuremath{|} @\coqdocvar{s\_nil} \coqdocvar{\_}\ensuremath{\Rightarrow} \coqdocvariable{s2}\coqdoceol
\coqdocindent{1.00em}
\ensuremath{|} \coqref{ex.ex2.s singleton}{\coqdocconstructor{s\_singleton}} \coqdocvar{a'} \coqdockw{as} \coqdocvar{a} \ensuremath{\Rightarrow} \coqref{ex.ex2.s cons}{\coqdocconstructor{s\_cons}} \coqdocvar{a} \coqdocvariable{s2}\coqdoceol
\coqdocindent{1.00em}
\ensuremath{|} \coqref{ex.ex2.s cons}{\coqdocconstructor{s\_cons}} \coqdocvar{a} \coqdocvar{s1'} \ensuremath{\Rightarrow} \coqref{ex.ex2.s cons}{\coqdocconstructor{s\_cons}} \coqdocvar{a} (\coqref{ex.s app}{\coqdocdefinition{s\_app}} \coqdocvar{s1'} \coqdocvariable{s2})\coqdoceol
\coqdocindent{1.00em}
\coqdockw{end}.\coqdoceol
\coqdocnoindent
\coqdockw{Lemma} \coqdef{ex.ex2.flattern distr}{flattern\_distr}{\coqdoclemma{flattern\_distr}} : \coqdockw{\ensuremath{\forall}} (\coqdocvar{X} : \coqdockw{Type}) (\coqdocvar{a} \coqdocvar{b} : \coqref{ex.ex2.slist}{\coqdocinductive{slist}} \coqdocvariable{X}), \coqref{ex.ex2.flattern}{\coqdocdefinition{flattern}} (\coqref{ex.ex2.s app}{\coqdocdefinition{s\_app}} \coqdocvariable{a} \coqdocvariable{b}) \coqexternalref{:type scope:x '=' x}{http://coq.inria.fr/distrib/8.5pl1/stdlib/Coq.Init.Logic}{\coqdocnotation{=}} \coqexternalref{:list scope:x '++' x}{http://coq.inria.fr/distrib/8.5pl1/stdlib/Coq.Init.Datatypes}{\coqdocnotation{(}}\coqref{ex.ex2.flattern}{\coqdocdefinition{flattern}} \coqdocvariable{a}\coqexternalref{:list scope:x '++' x}{http://coq.inria.fr/distrib/8.5pl1/stdlib/Coq.Init.Datatypes}{\coqdocnotation{)}} \coqexternalref{:list scope:x '++' x}{http://coq.inria.fr/distrib/8.5pl1/stdlib/Coq.Init.Datatypes}{\coqdocnotation{++}} \coqexternalref{:list scope:x '++' x}{http://coq.inria.fr/distrib/8.5pl1/stdlib/Coq.Init.Datatypes}{\coqdocnotation{(}}\coqref{ex.ex2.flattern}{\coqdocdefinition{flattern}} \coqdocvariable{b}\coqexternalref{:list scope:x '++' x}{http://coq.inria.fr/distrib/8.5pl1/stdlib/Coq.Init.Datatypes}{\coqdocnotation{)}}.\coqdoceol
\coqdocindent{1.00em}
\coqdoctac{induction} \coqdocvar{a}; \coqdoctac{intuition}.\coqdoceol
\coqdocindent{1.00em}
- \coqdoctac{simpl}. \coqdoctac{rewrite} \ensuremath{\leftarrow} \coqexternalref{app assoc}{http://coq.inria.fr/distrib/8.5pl1/stdlib/Coq.Lists.List}{\coqdoclemma{app\_assoc}}. \coqdoctac{rewrite} \ensuremath{\leftarrow} (\coqdocvar{IHa2} \coqdocvar{b}). \coqdoctac{reflexivity}. \coqdockw{Qed}.\coqdoceol
\coqdocnoindent
\coqdockw{End} \coqref{ex.ex2}{\coqdocmodule{ex2}}.\coqdoceol
\coqdocemptyline
\end{coqdoccode}
\item Modify the first example language of Chapter 2 to include variables, where variables are represented with \coqdocvar{nat}.  Extend the syntax and semantics of expressions to accommodate the change.  Your new \coqdocvar{expDenote} function should take as a new extra first argument a value of type \coqdocvar{var} \ensuremath{\rightarrow} \coqdocvar{nat}, where \coqdocvar{var} is a synonym for naturals-as-variables, and the function assigns a value to each variable.  Define a constant folding function which does a bottom-up pass over an expression, at each stage replacing every binary operation on constants with an equivalent constant.  Prove that constant folding preserves the meanings of expressions.
\begin{coqdoccode}
\coqdocemptyline
\coqdocnoindent
\coqdockw{Module} \coqdef{ex.ex3}{ex3}{\coqdocmodule{ex3}}.\coqdoceol
\coqdocnoindent
\coqdockw{Inductive} \coqdef{ex.ex3.binop}{binop}{\coqdocinductive{binop}} : \coqdockw{Set} := \coqdef{ex.ex3.Plus}{Plus}{\coqdocconstructor{Plus}} \ensuremath{|} \coqdef{ex.ex3.Times}{Times}{\coqdocconstructor{Times}}.\coqdoceol
\coqdocnoindent
\coqdockw{Inductive} \coqdef{ex.ex3.var}{var}{\coqdocinductive{var}} := \coqdef{ex.ex3.vvar}{vvar}{\coqdocconstructor{vvar}} : \coqexternalref{nat}{http://coq.inria.fr/distrib/8.5pl1/stdlib/Coq.Init.Datatypes}{\coqdocinductive{nat}}\coqexternalref{:type scope:x '->' x}{http://coq.inria.fr/distrib/8.5pl1/stdlib/Coq.Init.Logic}{\coqdocnotation{\ensuremath{\rightarrow}}} \coqref{ex.var}{\coqdocinductive{var}}.\coqdoceol
\coqdocnoindent
\coqdockw{Inductive} \coqdef{ex.ex3.exp}{exp}{\coqdocinductive{exp}} : \coqdockw{Set} :=\coqdoceol
\coqdocnoindent
\ensuremath{|} \coqdef{ex.ex3.Const}{Const}{\coqdocconstructor{Const}} : \coqexternalref{nat}{http://coq.inria.fr/distrib/8.5pl1/stdlib/Coq.Init.Datatypes}{\coqdocinductive{nat}} \coqexternalref{:type scope:x '->' x}{http://coq.inria.fr/distrib/8.5pl1/stdlib/Coq.Init.Logic}{\coqdocnotation{\ensuremath{\rightarrow}}} \coqref{ex.exp}{\coqdocinductive{exp}}\coqdoceol
\coqdocnoindent
\ensuremath{|} \coqdef{ex.ex3.Binop}{Binop}{\coqdocconstructor{Binop}} : \coqref{ex.ex3.binop}{\coqdocinductive{binop}} \coqexternalref{:type scope:x '->' x}{http://coq.inria.fr/distrib/8.5pl1/stdlib/Coq.Init.Logic}{\coqdocnotation{\ensuremath{\rightarrow}}} \coqref{ex.exp}{\coqdocinductive{exp}} \coqexternalref{:type scope:x '->' x}{http://coq.inria.fr/distrib/8.5pl1/stdlib/Coq.Init.Logic}{\coqdocnotation{\ensuremath{\rightarrow}}} \coqref{ex.exp}{\coqdocinductive{exp}} \coqexternalref{:type scope:x '->' x}{http://coq.inria.fr/distrib/8.5pl1/stdlib/Coq.Init.Logic}{\coqdocnotation{\ensuremath{\rightarrow}}} \coqref{ex.exp}{\coqdocinductive{exp}}\coqdoceol
\coqdocnoindent
\ensuremath{|} \coqdef{ex.ex3.Var}{Var}{\coqdocconstructor{Var}} : \coqref{ex.ex3.var}{\coqdocinductive{var}} \coqexternalref{:type scope:x '->' x}{http://coq.inria.fr/distrib/8.5pl1/stdlib/Coq.Init.Logic}{\coqdocnotation{\ensuremath{\rightarrow}}} \coqref{ex.exp}{\coqdocinductive{exp}}.\coqdoceol
\coqdocemptyline
\coqdocnoindent
\coqdockw{Definition} \coqdef{ex.ex3.binopDenote}{binopDenote}{\coqdocdefinition{binopDenote}} (\coqdocvar{b} : \coqref{ex.ex3.binop}{\coqdocinductive{binop}}) :=\coqdoceol
\coqdocnoindent
\coqdockw{match} \coqdocvariable{b} \coqdockw{with}\coqdoceol
\coqdocnoindent
\ensuremath{|} \coqref{ex.ex3.Plus}{\coqdocconstructor{Plus}} \ensuremath{\Rightarrow} \coqexternalref{plus}{http://coq.inria.fr/distrib/8.5pl1/stdlib/Coq.Init.Peano}{\coqdocabbreviation{plus}}\coqdoceol
\coqdocnoindent
\ensuremath{|} \coqref{ex.ex3.Times}{\coqdocconstructor{Times}} \ensuremath{\Rightarrow} \coqexternalref{mult}{http://coq.inria.fr/distrib/8.5pl1/stdlib/Coq.Init.Peano}{\coqdocabbreviation{mult}}\coqdoceol
\coqdocnoindent
\coqdockw{end}.\coqdoceol
\coqdocemptyline
\coqdocnoindent
\coqdockw{Fixpoint} \coqdef{ex.ex3.expDenote}{expDenote}{\coqdocdefinition{expDenote}} (\coqdocvar{ass} : \coqref{ex.ex3.var}{\coqdocinductive{var}}\coqexternalref{:type scope:x '->' x}{http://coq.inria.fr/distrib/8.5pl1/stdlib/Coq.Init.Logic}{\coqdocnotation{\ensuremath{\rightarrow}}} \coqexternalref{nat}{http://coq.inria.fr/distrib/8.5pl1/stdlib/Coq.Init.Datatypes}{\coqdocinductive{nat}}) (\coqdocvar{e} : \coqref{ex.ex3.exp}{\coqdocinductive{exp}}) : \coqexternalref{nat}{http://coq.inria.fr/distrib/8.5pl1/stdlib/Coq.Init.Datatypes}{\coqdocinductive{nat}} :=\coqdoceol
\coqdocindent{1.00em}
\coqdockw{match} \coqdocvariable{e} \coqdockw{with}\coqdoceol
\coqdocindent{1.00em}
\ensuremath{|} \coqref{ex.ex3.Const}{\coqdocconstructor{Const}} \coqdocvar{n} \ensuremath{\Rightarrow} \coqdocvar{n}\coqdoceol
\coqdocindent{1.00em}
\ensuremath{|} \coqref{ex.ex3.Binop}{\coqdocconstructor{Binop}} \coqdocvar{b} \coqdocvar{e1} \coqdocvar{e2} \ensuremath{\Rightarrow} (\coqref{ex.ex3.binopDenote}{\coqdocdefinition{binopDenote}} \coqdocvar{b}) (\coqref{ex.expDenote}{\coqdocdefinition{expDenote}} \coqdocvariable{ass} \coqdocvar{e1}) (\coqref{ex.expDenote}{\coqdocdefinition{expDenote}} \coqdocvariable{ass} \coqdocvar{e2})\coqdoceol
\coqdocindent{1.00em}
\ensuremath{|} \coqref{ex.ex3.Var}{\coqdocconstructor{Var}} \coqdocvar{v} \ensuremath{\Rightarrow} \coqdocvariable{ass} \coqdocvar{v}\coqdoceol
\coqdocindent{1.00em}
\coqdockw{end}.\coqdoceol
\coqdocemptyline
\coqdocnoindent
\coqdockw{Fixpoint} \coqdef{ex.ex3.const fold}{const\_fold}{\coqdocdefinition{const\_fold}} (\coqdocvar{e} : \coqref{ex.ex3.exp}{\coqdocinductive{exp}}) : \coqref{ex.ex3.exp}{\coqdocinductive{exp}}:=\coqdoceol
\coqdocindent{1.00em}
\coqdockw{match} \coqdocvariable{e} \coqdockw{with}\coqdoceol
\coqdocindent{1.00em}
\ensuremath{|} \coqref{ex.ex3.Const}{\coqdocconstructor{Const}} \coqdocvar{n} \ensuremath{\Rightarrow} \coqdocvariable{e}\coqdoceol
\coqdocindent{1.00em}
\ensuremath{|} \coqref{ex.ex3.Var}{\coqdocconstructor{Var}} \coqdocvar{v} \ensuremath{\Rightarrow} \coqdocvariable{e}\coqdoceol
\coqdocindent{1.00em}
\ensuremath{|} \coqref{ex.ex3.Binop}{\coqdocconstructor{Binop}} \coqdocvar{b} \coqdocvar{e1} \coqdocvar{e2} \ensuremath{\Rightarrow} \coqdockw{match} \coqdocvar{e1}, \coqdocvar{e2} \coqdockw{with}\coqdoceol
\coqdocindent{10.00em}
\ensuremath{|} \coqref{ex.ex3.Const}{\coqdocconstructor{Const}} \coqdocvar{n1}, \coqref{ex.ex3.Const}{\coqdocconstructor{Const}} \coqdocvar{n2} \ensuremath{\Rightarrow} \coqref{ex.ex3.Const}{\coqdocconstructor{Const}} ((\coqref{ex.ex3.binopDenote}{\coqdocdefinition{binopDenote}} \coqdocvar{b}) \coqdocvar{n1} \coqdocvar{n2})\coqdoceol
\coqdocindent{10.00em}
\ensuremath{|} \coqdocvar{\_}, \coqdocvar{\_} \ensuremath{\Rightarrow} \coqref{ex.ex3.Binop}{\coqdocconstructor{Binop}} \coqdocvar{b} (\coqref{ex.const fold}{\coqdocdefinition{const\_fold}} \coqdocvar{e1}) (\coqref{ex.const fold}{\coqdocdefinition{const\_fold}} \coqdocvar{e2})\coqdoceol
\coqdocindent{10.00em}
\coqdockw{end}\coqdoceol
\coqdocindent{1.00em}
\coqdockw{end}.\coqdoceol
\coqdocemptyline
\coqdocnoindent
\coqdockw{Lemma} \coqdef{ex.ex3.const fold correct}{const\_fold\_correct}{\coqdoclemma{const\_fold\_correct}} : \coqdockw{\ensuremath{\forall}} (\coqdocvar{e} : \coqref{ex.ex3.exp}{\coqdocinductive{exp}}) (\coqdocvar{ass} : \coqref{ex.ex3.var}{\coqdocinductive{var}} \coqexternalref{:type scope:x '->' x}{http://coq.inria.fr/distrib/8.5pl1/stdlib/Coq.Init.Logic}{\coqdocnotation{\ensuremath{\rightarrow}}} \coqexternalref{nat}{http://coq.inria.fr/distrib/8.5pl1/stdlib/Coq.Init.Datatypes}{\coqdocinductive{nat}}), \coqref{ex.ex3.expDenote}{\coqdocdefinition{expDenote}} \coqdocvariable{ass} \coqdocvariable{e} \coqexternalref{:type scope:x '=' x}{http://coq.inria.fr/distrib/8.5pl1/stdlib/Coq.Init.Logic}{\coqdocnotation{=}} \coqref{ex.ex3.expDenote}{\coqdocdefinition{expDenote}} \coqdocvariable{ass} (\coqref{ex.ex3.const fold}{\coqdocdefinition{const\_fold}} \coqdocvariable{e}).\coqdoceol
\coqdocindent{1.00em}
\coqdoctac{induction} \coqdocvar{e}; \coqdoctac{intuition}; \coqdoctac{destruct} \coqdocvar{b}; \coqdoctac{induction} \coqdocvar{e1}, \coqdocvar{e2};\coqdoceol
\coqdocindent{2.00em}
\coqdoctac{auto}; \coqdoctac{simpl}; \coqdoctac{f\_equal}; \coqdoctac{simpl} \coqdoctac{in} *; \coqdoctac{auto}. \coqdockw{Qed}.\coqdoceol
\coqdocnoindent
\coqdockw{End} \coqref{ex.ex3}{\coqdocmodule{ex3}}.\coqdoceol
\coqdocemptyline
\end{coqdoccode}
\item Reimplement the second example language of Chapter 2 to use mutually inductive types instead of dependent types.  That is, define two separate (non-dependent) inductive types \coqdocvar{nat\_exp} and \coqdocvar{bool\_exp} for expressions of the two different types, rather than a single indexed type.  To keep things simple, you may consider only the binary operators that take naturals as operands.  Add natural number variables to the language, as in the last exercise, and add an ``if'' expression form taking as arguments one boolean expression and two natural number expressions.  Define semantics and constant-folding functions for this new language.  Your constant folding should simplify not just binary operations (returning naturals or booleans) with known arguments, but also ``if'' expressions with known values for their test expressions but possibly undetermined ``then'' and ``else'' cases.  Prove that constant-folding a natural number expression preserves its meaning.
 \begin{coqdoccode}
\coqdocemptyline
\coqdocnoindent
\coqdockw{Module} \coqdef{ex.ex4}{ex4}{\coqdocmodule{ex4}}.\coqdoceol
\coqdocindent{1.00em}
\coqdockw{Require} \coqdockw{Import} \coqexternalref{}{http://coq.inria.fr/distrib/8.5pl1/stdlib/Coq.Arith.Arith}{\coqdoclibrary{Arith}}.\coqdoceol
\coqdocindent{1.00em}
\coqdockw{Inductive} \coqdef{ex.ex4.nbinop}{nbinop}{\coqdocinductive{nbinop}} : \coqdockw{Set} := \coqdef{ex.ex4.NPlus}{NPlus}{\coqdocconstructor{NPlus}} \ensuremath{|} \coqdef{ex.ex4.NTimes}{NTimes}{\coqdocconstructor{NTimes}}.\coqdoceol
\coqdocindent{1.00em}
\coqdockw{Inductive} \coqdef{ex.ex4.bbinop}{bbinop}{\coqdocinductive{bbinop}} : \coqdockw{Set} := \coqdef{ex.ex4.TEq}{TEq}{\coqdocconstructor{TEq}} \ensuremath{|} \coqdef{ex.ex4.TLt}{TLt}{\coqdocconstructor{TLt}}.\coqdoceol
\coqdocemptyline
\coqdocindent{1.00em}
\coqdockw{Inductive} \coqdef{ex.ex4.var}{var}{\coqdocinductive{var}} : \coqdockw{Set} := \coqdef{ex.ex4.vvar}{vvar}{\coqdocconstructor{vvar}} : \coqexternalref{nat}{http://coq.inria.fr/distrib/8.5pl1/stdlib/Coq.Init.Datatypes}{\coqdocinductive{nat}} \coqexternalref{:type scope:x '->' x}{http://coq.inria.fr/distrib/8.5pl1/stdlib/Coq.Init.Logic}{\coqdocnotation{\ensuremath{\rightarrow}}} \coqref{ex.var}{\coqdocinductive{var}}.\coqdoceol
\coqdocindent{1.00em}
\coqdockw{Inductive} \coqdef{ex.ex4.bool exp}{bool\_exp}{\coqdocinductive{bool\_exp}} : \coqdockw{Set}:=\coqdoceol
\coqdocindent{1.00em}
\ensuremath{|} \coqdef{ex.ex4.BEq}{BEq}{\coqdocconstructor{BEq}} : \coqexternalref{nat}{http://coq.inria.fr/distrib/8.5pl1/stdlib/Coq.Init.Datatypes}{\coqdocinductive{nat}} \coqexternalref{:type scope:x '->' x}{http://coq.inria.fr/distrib/8.5pl1/stdlib/Coq.Init.Logic}{\coqdocnotation{\ensuremath{\rightarrow}}} \coqexternalref{nat}{http://coq.inria.fr/distrib/8.5pl1/stdlib/Coq.Init.Datatypes}{\coqdocinductive{nat}} \coqexternalref{:type scope:x '->' x}{http://coq.inria.fr/distrib/8.5pl1/stdlib/Coq.Init.Logic}{\coqdocnotation{\ensuremath{\rightarrow}}} \coqref{ex.bool exp}{\coqdocinductive{bool\_exp}}\coqdoceol
\coqdocindent{1.00em}
\ensuremath{|} \coqdef{ex.ex4.BLt}{BLt}{\coqdocconstructor{BLt}} : \coqexternalref{nat}{http://coq.inria.fr/distrib/8.5pl1/stdlib/Coq.Init.Datatypes}{\coqdocinductive{nat}} \coqexternalref{:type scope:x '->' x}{http://coq.inria.fr/distrib/8.5pl1/stdlib/Coq.Init.Logic}{\coqdocnotation{\ensuremath{\rightarrow}}} \coqexternalref{nat}{http://coq.inria.fr/distrib/8.5pl1/stdlib/Coq.Init.Datatypes}{\coqdocinductive{nat}} \coqexternalref{:type scope:x '->' x}{http://coq.inria.fr/distrib/8.5pl1/stdlib/Coq.Init.Logic}{\coqdocnotation{\ensuremath{\rightarrow}}} \coqref{ex.bool exp}{\coqdocinductive{bool\_exp}}\coqdoceol
\coqdocindent{1.00em}
\ensuremath{|} \coqdef{ex.ex4.BConst}{BConst}{\coqdocconstructor{BConst}} : \coqexternalref{bool}{http://coq.inria.fr/distrib/8.5pl1/stdlib/Coq.Init.Datatypes}{\coqdocinductive{bool}} \coqexternalref{:type scope:x '->' x}{http://coq.inria.fr/distrib/8.5pl1/stdlib/Coq.Init.Logic}{\coqdocnotation{\ensuremath{\rightarrow}}} \coqref{ex.bool exp}{\coqdocinductive{bool\_exp}}.\coqdoceol
\coqdocemptyline
\coqdocindent{1.00em}
\coqdockw{Inductive} \coqdef{ex.ex4.nat exp}{nat\_exp}{\coqdocinductive{nat\_exp}} : \coqdockw{Set} :=\coqdoceol
\coqdocindent{1.00em}
\ensuremath{|} \coqdef{ex.ex4.NConst}{NConst}{\coqdocconstructor{NConst}} : \coqexternalref{nat}{http://coq.inria.fr/distrib/8.5pl1/stdlib/Coq.Init.Datatypes}{\coqdocinductive{nat}} \coqexternalref{:type scope:x '->' x}{http://coq.inria.fr/distrib/8.5pl1/stdlib/Coq.Init.Logic}{\coqdocnotation{\ensuremath{\rightarrow}}} \coqref{ex.nat exp}{\coqdocinductive{nat\_exp}}\coqdoceol
\coqdocindent{1.00em}
\ensuremath{|} \coqdef{ex.ex4.NBinop}{NBinop}{\coqdocconstructor{NBinop}} : \coqref{ex.ex4.nbinop}{\coqdocinductive{nbinop}} \coqexternalref{:type scope:x '->' x}{http://coq.inria.fr/distrib/8.5pl1/stdlib/Coq.Init.Logic}{\coqdocnotation{\ensuremath{\rightarrow}}} \coqref{ex.nat exp}{\coqdocinductive{nat\_exp}} \coqexternalref{:type scope:x '->' x}{http://coq.inria.fr/distrib/8.5pl1/stdlib/Coq.Init.Logic}{\coqdocnotation{\ensuremath{\rightarrow}}} \coqref{ex.nat exp}{\coqdocinductive{nat\_exp}} \coqexternalref{:type scope:x '->' x}{http://coq.inria.fr/distrib/8.5pl1/stdlib/Coq.Init.Logic}{\coqdocnotation{\ensuremath{\rightarrow}}} \coqref{ex.nat exp}{\coqdocinductive{nat\_exp}}\coqdoceol
\coqdocindent{1.00em}
\ensuremath{|} \coqdef{ex.ex4.NVar}{NVar}{\coqdocconstructor{NVar}} : \coqref{ex.ex4.var}{\coqdocinductive{var}} \coqexternalref{:type scope:x '->' x}{http://coq.inria.fr/distrib/8.5pl1/stdlib/Coq.Init.Logic}{\coqdocnotation{\ensuremath{\rightarrow}}} \coqref{ex.nat exp}{\coqdocinductive{nat\_exp}}\coqdoceol
\coqdocindent{1.00em}
\ensuremath{|} \coqdef{ex.ex4.NIf}{NIf}{\coqdocconstructor{NIf}} : \coqref{ex.ex4.bool exp}{\coqdocinductive{bool\_exp}} \coqexternalref{:type scope:x '->' x}{http://coq.inria.fr/distrib/8.5pl1/stdlib/Coq.Init.Logic}{\coqdocnotation{\ensuremath{\rightarrow}}} \coqref{ex.nat exp}{\coqdocinductive{nat\_exp}} \coqexternalref{:type scope:x '->' x}{http://coq.inria.fr/distrib/8.5pl1/stdlib/Coq.Init.Logic}{\coqdocnotation{\ensuremath{\rightarrow}}} \coqref{ex.nat exp}{\coqdocinductive{nat\_exp}} \coqexternalref{:type scope:x '->' x}{http://coq.inria.fr/distrib/8.5pl1/stdlib/Coq.Init.Logic}{\coqdocnotation{\ensuremath{\rightarrow}}} \coqref{ex.nat exp}{\coqdocinductive{nat\_exp}}.\coqdoceol
\coqdocemptyline
\coqdocindent{1.00em}
\coqdockw{Definition} \coqdef{ex.ex4.bbinopDenote}{bbinopDenote}{\coqdocdefinition{bbinopDenote}} \coqdocvar{bb} :=\coqdoceol
\coqdocindent{1.00em}
\coqdockw{match} \coqdocvariable{bb} \coqdockw{with}\coqdoceol
\coqdocindent{1.00em}
\ensuremath{|} \coqref{ex.ex4.TEq}{\coqdocconstructor{TEq}} \ensuremath{\Rightarrow} \coqexternalref{beq nat}{http://coq.inria.fr/distrib/8.5pl1/stdlib/Coq.Arith.EqNat}{\coqdocabbreviation{beq\_nat}}\coqdoceol
\coqdocindent{1.00em}
\ensuremath{|} \coqref{ex.ex4.TLt}{\coqdocconstructor{TLt}} \ensuremath{\Rightarrow} \coqexternalref{Nat.leb}{http://coq.inria.fr/distrib/8.5pl1/stdlib/Coq.Arith.PeanoNat}{\coqdocdefinition{Nat.leb}}\coqdoceol
\coqdocindent{1.00em}
\coqdockw{end}.\coqdoceol
\coqdocemptyline
\coqdocindent{1.00em}
\coqdockw{Definition} \coqdef{ex.ex4.nbinopDenote}{nbinopDenote}{\coqdocdefinition{nbinopDenote}} \coqdocvar{nb} :=\coqdoceol
\coqdocindent{1.00em}
\coqdockw{match} \coqdocvariable{nb} \coqdockw{with}\coqdoceol
\coqdocindent{1.00em}
\ensuremath{|} \coqref{ex.ex4.NPlus}{\coqdocconstructor{NPlus}} \ensuremath{\Rightarrow} \coqexternalref{plus}{http://coq.inria.fr/distrib/8.5pl1/stdlib/Coq.Init.Peano}{\coqdocabbreviation{plus}}\coqdoceol
\coqdocindent{1.00em}
\ensuremath{|} \coqref{ex.ex4.NTimes}{\coqdocconstructor{NTimes}} \ensuremath{\Rightarrow} \coqexternalref{mult}{http://coq.inria.fr/distrib/8.5pl1/stdlib/Coq.Init.Peano}{\coqdocabbreviation{mult}}\coqdoceol
\coqdocindent{1.00em}
\coqdockw{end}.\coqdoceol
\coqdocemptyline
\coqdocindent{1.00em}
\coqdockw{Fixpoint} \coqdef{ex.ex4.bexpDenote}{bexpDenote}{\coqdocdefinition{bexpDenote}} (\coqdocvar{e} : \coqref{ex.ex4.bool exp}{\coqdocinductive{bool\_exp}}) : \coqexternalref{bool}{http://coq.inria.fr/distrib/8.5pl1/stdlib/Coq.Init.Datatypes}{\coqdocinductive{bool}} :=\coqdoceol
\coqdocindent{2.00em}
\coqdockw{match} \coqdocvariable{e} \coqdockw{with}\coqdoceol
\coqdocindent{2.00em}
\ensuremath{|} \coqref{ex.ex4.BEq}{\coqdocconstructor{BEq}} \coqdocvar{n1} \coqdocvar{n2} \ensuremath{\Rightarrow} \coqexternalref{beq nat}{http://coq.inria.fr/distrib/8.5pl1/stdlib/Coq.Arith.EqNat}{\coqdocabbreviation{beq\_nat}} \coqdocvar{n1} \coqdocvar{n2}\coqdoceol
\coqdocindent{2.00em}
\ensuremath{|} \coqref{ex.ex4.BLt}{\coqdocconstructor{BLt}} \coqdocvar{n1} \coqdocvar{n2} \ensuremath{\Rightarrow} \coqexternalref{Nat.leb}{http://coq.inria.fr/distrib/8.5pl1/stdlib/Coq.Arith.PeanoNat}{\coqdocdefinition{Nat.leb}} \coqdocvar{n1} \coqdocvar{n2}\coqdoceol
\coqdocindent{2.00em}
\ensuremath{|} \coqref{ex.ex4.BConst}{\coqdocconstructor{BConst}} \coqdocvar{b} \ensuremath{\Rightarrow} \coqdocvar{b}\coqdoceol
\coqdocindent{2.00em}
\coqdockw{end}.\coqdoceol
\coqdocemptyline
\coqdocindent{1.00em}
\coqdockw{Fixpoint} \coqdef{ex.ex4.nexpDenote}{nexpDenote}{\coqdocdefinition{nexpDenote}} (\coqdocvar{ass} : \coqref{ex.ex4.var}{\coqdocinductive{var}} \coqexternalref{:type scope:x '->' x}{http://coq.inria.fr/distrib/8.5pl1/stdlib/Coq.Init.Logic}{\coqdocnotation{\ensuremath{\rightarrow}}} \coqexternalref{nat}{http://coq.inria.fr/distrib/8.5pl1/stdlib/Coq.Init.Datatypes}{\coqdocinductive{nat}}) (\coqdocvar{e} : \coqref{ex.ex4.nat exp}{\coqdocinductive{nat\_exp}}) : \coqexternalref{nat}{http://coq.inria.fr/distrib/8.5pl1/stdlib/Coq.Init.Datatypes}{\coqdocinductive{nat}} :=\coqdoceol
\coqdocindent{2.00em}
\coqdockw{match} \coqdocvariable{e} \coqdockw{with}\coqdoceol
\coqdocindent{2.00em}
\ensuremath{|} \coqref{ex.ex4.NConst}{\coqdocconstructor{NConst}} \coqdocvar{n1} \ensuremath{\Rightarrow} \coqdocvar{n1}\coqdoceol
\coqdocindent{2.00em}
\ensuremath{|} \coqref{ex.ex4.NBinop}{\coqdocconstructor{NBinop}} \coqdocvar{b} \coqdocvar{e1} \coqdocvar{e2} \ensuremath{\Rightarrow} (\coqref{ex.ex4.nbinopDenote}{\coqdocdefinition{nbinopDenote}} \coqdocvar{b}) (\coqref{ex.nexpDenote}{\coqdocdefinition{nexpDenote}} \coqdocvariable{ass} \coqdocvar{e1}) (\coqref{ex.nexpDenote}{\coqdocdefinition{nexpDenote}} \coqdocvariable{ass} \coqdocvar{e2})\coqdoceol
\coqdocindent{2.00em}
\ensuremath{|} \coqref{ex.ex4.NVar}{\coqdocconstructor{NVar}} \coqdocvar{v} \ensuremath{\Rightarrow} \coqdocvariable{ass} \coqdocvar{v}\coqdoceol
\coqdocindent{2.00em}
\ensuremath{|} \coqref{ex.ex4.NIf}{\coqdocconstructor{NIf}} \coqdocvar{b} \coqdocvar{e1} \coqdocvar{e2} \ensuremath{\Rightarrow} \coqdockw{if} (\coqref{ex.ex4.bexpDenote}{\coqdocdefinition{bexpDenote}} \coqdocvar{b}) \coqdockw{then} (\coqref{ex.nexpDenote}{\coqdocdefinition{nexpDenote}} \coqdocvariable{ass} \coqdocvar{e1}) \coqdockw{else} (\coqref{ex.nexpDenote}{\coqdocdefinition{nexpDenote}} \coqdocvariable{ass} \coqdocvar{e2})\coqdoceol
\coqdocindent{2.00em}
\coqdockw{end}.\coqdoceol
\coqdocemptyline
\coqdocindent{1.00em}
\coqdockw{Fixpoint} \coqdef{ex.ex4.fold const}{fold\_const}{\coqdocdefinition{fold\_const}} (\coqdocvar{e} : \coqref{ex.ex4.nat exp}{\coqdocinductive{nat\_exp}}) : \coqref{ex.ex4.nat exp}{\coqdocinductive{nat\_exp}} :=\coqdoceol
\coqdocindent{2.00em}
\coqdockw{match} \coqdocvariable{e} \coqdockw{with}\coqdoceol
\coqdocindent{2.00em}
\ensuremath{|} \coqref{ex.ex4.NBinop}{\coqdocconstructor{NBinop}} \coqdocvar{b} \coqdocvar{e1} \coqdocvar{e2} \ensuremath{\Rightarrow} \coqdockw{match} \coqdocvar{e1}, \coqdocvar{e2} \coqdockw{with}\coqdoceol
\coqdocindent{11.50em}
\ensuremath{|} \coqref{ex.ex4.NConst}{\coqdocconstructor{NConst}} \coqdocvar{n1}, \coqref{ex.ex4.NConst}{\coqdocconstructor{NConst}} \coqdocvar{n2} \ensuremath{\Rightarrow} \coqref{ex.ex4.NConst}{\coqdocconstructor{NConst}} ((\coqref{ex.ex4.nbinopDenote}{\coqdocdefinition{nbinopDenote}} \coqdocvar{b}) \coqdocvar{n1} \coqdocvar{n2})\coqdoceol
\coqdocindent{11.50em}
\ensuremath{|} \coqdocvar{\_}, \coqdocvar{\_} \ensuremath{\Rightarrow} \coqref{ex.ex4.NBinop}{\coqdocconstructor{NBinop}} \coqdocvar{b} (\coqref{ex.fold const}{\coqdocdefinition{fold\_const}} \coqdocvar{e1}) (\coqref{ex.fold const}{\coqdocdefinition{fold\_const}} \coqdocvar{e2})\coqdoceol
\coqdocindent{11.50em}
\coqdockw{end}\coqdoceol
\coqdocindent{2.00em}
\ensuremath{|} \coqref{ex.ex4.NIf}{\coqdocconstructor{NIf}} \coqdocvar{b} \coqdocvar{e1} \coqdocvar{e2} \ensuremath{\Rightarrow} \coqdockw{if} (\coqref{ex.ex4.bexpDenote}{\coqdocdefinition{bexpDenote}} \coqdocvar{b}) \coqdockw{then} (\coqref{ex.fold const}{\coqdocdefinition{fold\_const}} \coqdocvar{e1}) \coqdockw{else} (\coqref{ex.fold const}{\coqdocdefinition{fold\_const}} \coqdocvar{e2})\coqdoceol
\coqdocindent{2.00em}
\ensuremath{|} \coqdocvar{\_} \ensuremath{\Rightarrow} \coqdocvariable{e}\coqdoceol
\coqdocindent{2.00em}
\coqdockw{end}.\coqdoceol
\coqdocemptyline
\coqdocindent{1.00em}
\coqdockw{Lemma} \coqdef{ex.ex4.fold const correct}{fold\_const\_correct}{\coqdoclemma{fold\_const\_correct}} : \coqdockw{\ensuremath{\forall}} (\coqdocvar{e} : \coqref{ex.ex4.nat exp}{\coqdocinductive{nat\_exp}}) (\coqdocvar{ass} : \coqref{ex.ex4.var}{\coqdocinductive{var}}\coqexternalref{:type scope:x '->' x}{http://coq.inria.fr/distrib/8.5pl1/stdlib/Coq.Init.Logic}{\coqdocnotation{\ensuremath{\rightarrow}}} \coqexternalref{nat}{http://coq.inria.fr/distrib/8.5pl1/stdlib/Coq.Init.Datatypes}{\coqdocinductive{nat}}),\coqdoceol
\coqdocindent{3.00em}
\coqref{ex.ex4.nexpDenote}{\coqdocdefinition{nexpDenote}} \coqdocvariable{ass} \coqdocvariable{e} \coqexternalref{:type scope:x '=' x}{http://coq.inria.fr/distrib/8.5pl1/stdlib/Coq.Init.Logic}{\coqdocnotation{=}} \coqref{ex.ex4.nexpDenote}{\coqdocdefinition{nexpDenote}} \coqdocvariable{ass} (\coqref{ex.ex4.fold const}{\coqdocdefinition{fold\_const}} \coqdocvariable{e}).\coqdoceol
\coqdocindent{2.00em}
\coqdoctac{induction} \coqdocvar{e}; \coqdoctac{intuition}.\coqdoceol
\coqdocindent{2.00em}
- \coqdoctac{destruct} \coqdocvar{n}; \coqdoctac{induction} \coqdocvar{e1}, \coqdocvar{e2}; \coqdoctac{auto}; \coqdoctac{simpl}; \coqdoctac{f\_equal}; \coqdoctac{simpl} \coqdoctac{in} *; \coqdoctac{auto}.\coqdoceol
\coqdocindent{2.00em}
- \coqdoctac{destruct} \coqdocvar{b}; \coqdoctac{try} \coqdoctac{destruct} \coqdocvar{b}; \coqdoctac{auto}; \coqdoctac{simpl}; \coqdoctac{intuition}; \coqdoctac{destruct} (\coqdocvar{n} \coqexternalref{:nat scope:x '<=?' x}{http://coq.inria.fr/distrib/8.5pl1/stdlib/Coq.Arith.PeanoNat}{\coqdocnotation{<=?}} \coqdocvar{n0});\coqdoceol
\coqdocindent{4.00em}
\coqdoctac{destruct} (\coqdocvar{n} \coqexternalref{:nat scope:x '=?' x}{http://coq.inria.fr/distrib/8.5pl1/stdlib/Coq.Arith.PeanoNat}{\coqdocnotation{=?}} \coqdocvar{n0}); \coqdoctac{auto}. \coqdockw{Qed}.\coqdoceol
\coqdocnoindent
\coqdockw{End} \coqref{ex.ex4}{\coqdocmodule{ex4}}.\coqdoceol
\end{coqdoccode}
\item Define mutually inductive types of even and odd natural numbers, such that any natural number is isomorphic to a value of one of the two types.  (This problem does not ask you to prove that correspondence, though some interpretations of the task may be interesting exercises.)  Write a function that computes the sum of two even numbers, such that the function type guarantees that the output is even as well.  Prove that this function is commutative.
 \begin{coqdoccode}
\coqdocemptyline
\coqdocnoindent
\coqdockw{Module} \coqdef{ex.ex5}{ex5}{\coqdocmodule{ex5}}.\coqdoceol
\coqdocemptyline
\coqdocnoindent
\coqdockw{End} \coqref{ex.ex5}{\coqdocmodule{ex5}}.\coqdoceol
\coqdocemptyline
\end{coqdoccode}
\end{enumerate} \begin{coqdoccode}
\end{coqdoccode}
\end{document}
